\documentclass[ngerman]{report}

\usepackage{../style}
\usepackage{fancyhdr}
\usepackage{tikz-cd}
\title{Einführung in die Funktionentheorie und Gewöhnliche Differentialgleichungen SS24}
\author{Raphael Frey, Benni Kleisz}
\institute{Universität Tübingen}
\department{Mathematik}
\date{\today}


\fancypagestyle{myfancystyle}{
  \fancyhf{}
  \fancyhead[L]{
    \begin{minipage}{1em}
        \rule[-1em]{1pt}{3em}
    \end{minipage}
    \begin{minipage}{\textwidth-2em}
        \emph{Aufgaben Funktionentheorie und Gewöhnliche Differentialgleichungen}\\
        \textbf{Sommersemester 2024 Blatt 6}\\
        Raphael Frey, Benni Kleisz, Leonie Kempf
    \end{minipage}
    }
  \renewcommand{\headrulewidth}{0pt}
  \renewcommand{\footrulewidth}{0pt}
}
\setlength{\headsep}{4em}
\pagestyle{myfancystyle}

\begin{document}
\setcounter{question}{1}
\setcounter{answer}{1}
\begin{question}
    Sei $U \subset \mathbb{C}$ offen, $h : U \to \mathbb{C}$ sei 2-mal stetig partiell differenzierbar nach $x$ und nach $y$. Erinnerung: $h$ ist harmonisch, wenn mit dem Laplace-Operator $\laplace := \pdv[2]{x} + \pdv[2]{y}$ gilt: $\laplace h = 0$. Zeigen Sie:
\begin{enumerate}
    \item Für jedes 2-mal stetig partiell differenzierbare $h$
    \begin{align*}
        \overline{\pdv{h}{z}} = \pdv{\bar{h}}{\bar{z}} \quad \text{und} \quad \overline{\pdv{h}{\bar{z}}} = \pdv{\bar{h}}{z} \quad \text{und}\quad \laplace h = 4 \pdv{z \partial \bar{z}} h = 4 \frac{\partial^2}{\partial \bar{z} \partial z} h
    \end{align*}
    (mit $ \pdv{z} = \frac{1}{2} \left( \pdv{x} - i \pdv{y} \right) $ und $ \pdv{\bar{z}} = \frac{1}{2} \left( \pdv{x} + i \pdv{y} \right) $ und Ansatz $ h = u + iv $. Diffquotient o.Ä. nicht nötig).
    
    \item Wenn $ f \in \mc{O}(U) $, dann ist $ f $ harmonisch. Wann ist $ \overline{f} $ (bzw. $ \Re f $ bzw. $ \Im f $) holomorph?
    \item Wenn $ h $ harmonisch auf $ U $, dann sind $ h, \Re h, \Im h $ harmonisch. Ist $ h $ stets holomorph?
    \item Wenn $ h $ harmonisch auf $ U $ und $ U $ einfach zusammenhängend, dann $ h = f + \overline{g} $ für $ f, g \in \mc{O}(U) $. Wenn $ h $ reellwertig, dann $ h = \Re f $ für ein $ f \in \mc{O}(U) $.
    \item Ein auf $ U $ harmonisches $ h $ genügt der Mittelwerteigenschaft
    \[
    h(p) = \frac{1}{2\pi} \int_0^{2\pi} h(p + re^{it}) \, dt \quad (\overline{D_r(p)} \subset U).
    \]
    (vgl. Bemerkung zu Cauchys Integralformel).
    
    (Bem: Wegen 4. $ h $ harmonisch $\Rightarrow h \in C^\infty $ und analytisch, lokal $ \sum c_k (z - p)^k + \sum d_k (z - p)^k $. Aufwändiger: $ h : U \to \mathbb{C} $ Mittelwerteigenschaft $\Rightarrow h $ harmonisch. Viele Anwendungen in Physik, bei partiellen DGL).
\end{enumerate}
\end{question}
\begin{answer}
    \begin{enumerate}
        \setcounter{enumi}{3}
        \item Nach Aufgabenteil 1 gilt
        \begin{align*}
            0 = \laplace h \implies \pder[h]{z} \;\text{holomorph}.
        \end{align*}
        Da $U$ einfach zusammenhängend ist, ist die Stammfunktion ($=h$ bis auf eine Konstante o.B.d.A. zu $h$ über hilfsfunktion umzuschreiben) auch holomorph. Es folgt, dass $h$ holomorph ist....??? (keine Ahnung was hier falsch läuft)
        \item 
    \end{enumerate}
\end{answer}
\newpage
\begin{question}
    Nach VL (Bsp. 1.4) ist das AWP
    \begin{align*}
        y' = \sqrt{\abs{y}}\\
        y(0) = 0
    \end{align*}
    nicht eindeutig lösbar. Welche Lösung wird durch den Satz von Peano gefunden? Begründen Sie Ihre Antwort. Zeigen Sie, dass $f (y) = \sqrt{|y|}$ in keiner Umgebung von $y = 0$ Lipschitz-stetig ist.
\end{question}
\begin{answer}
    O.B.d.A. Sei die Umgebung durch ein Intervall $I$ gegeben. Auf $I\setminus\{0\}$ ist $f$ stetig differenzierbar mit 
    \begin{align*}
        f'(y) = \frac{1}{2\sqrt{|y|}}.
    \end{align*}
    Falls $f$ stetig differenzierbar (auf einer dichten Teilmenge) ist, so gilt für die Lipschitz-Konstante
    \begin{align*}
        L \geq \sup_{y\in I} \abs{f'(y)} = \infty.
    \end{align*}
    Also ist $f$ nicht Lipschitz-stetig. Der Satz von Peano besagt, dass eine Lösung existiert, welche mithilfe des Eulerverfahrens gefunden werden kann. Es gilt 
    \begin{align*}
        y(t+h) = y(t) + h f'(y(t)) + \mathcal{O}(h^2).
    \end{align*}
    mit $t_0 = 0$ gilt
    \begin{align*}
        y(h) = h f'(0) = h\cdot 0 = 0.
    \end{align*} 
\end{answer}
\end{document}